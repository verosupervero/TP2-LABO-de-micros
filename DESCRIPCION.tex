\section{Descripción del proyecto}\label{sec:intro}
\par Se conectarán el display de 7 segmentos, los dos pulsadores y el Arduino UNO de acuerdo a como se indica en el esquemático. El punto decimal del display no se utilizará.

\par Cada segmento del display es un LED, que como sabemos tienen polaridad. Según el modelo de display, los cátodos de todos los LEDs están conectados a GND y por lo tanto los LEDs se encenderán sólo al colocar un “1” lógico.
\par La elección de qué pines se usaron para la conexión puede también observarse en el esquemático.
\par Respecto a los pulsadores, estarán conectados a los pines PD2 y PD3, que en el microcontrolador Atmega328 se vinculan a las interrupciones externas INT0 e INT1.
Dado que esta vinculación es fija, la posición de estos pulsadores no puede modificarse. Cuando el pulsador está abierto, el pin , ya sea PD2 como PD3, queda circuitalmente “flotando”, es decir, no es ni un “1” ni un “0” lógico, circunstancia que como es sabido debe evitarse. Se abordará ese inconveniente.
\par El programa a implementar hará lo siguiente:
 \par - El display inicialmente mostrará el dígito “5”.
\par - Si al presionar el pulsador conectado a PD2, el pulsador conectado a PD3 también estuviera presionado, entonces se incrementará el dígito decimal, siempre que no se supere el valor máximo, es decir “9”. Si en cambio, el pulsador conectado a PD3 no hubiera estado presionado al momento de presionar PD2, se deberá decrementar el valor decimal mostrado, siempre que no se haya alcanzado el mínimo, es decir, un “0”.
\par En resumen, PD2 ordena un cambio de dígito y PD3 provee el sentido (incremento o decremento).
\par  El manejo de PD2 será usando interrupciones, modalidad de flanco, en tanto el manejo de PD3 será por lectura explícita del valor instantáneo del pin.
Se evitará que al presionar PD2 el display avance más de un dígito, existen dos formas posibles: una es realizando un delay y la segunda es colocando un capacitor indicado. Esta circunstancia indeseada se llama “rebote” y está vinculada al ruido originado en la mecánica del pulsador.
\par Los valores de los siete segmentos de cada uno de los diez símbolos decimales están puestos en una tabla a ser almacenada en la memoria del programa.